On ne se nourrit pas seulement de plats plus ou moins sophistiqués mais aussi de souvenirs… Quoi de mieux que les odeurs et le goût de la cuisine de notre enfance pour revivre des moments agréables. Pourquoi le parfum de la viande qui crépite dans le beurre au fond de la cocotte avant de cuire en ragoût fait remonter à la mémoire le goût du café au lait ?$\ldots$ Parce que le jeudi, jour où il n’y avait pas d’école, on se levait plus tard et on prenait son petit déjeuner dans la cuisine où maman commençait déjà la préparation du repas de midi$\ldots$
Dans notre famille, chaque génération a laissé son empreinte dans la tradition culinaire. Elle s’est construite à partir des habitudes gastronomiques, des voyages et des rencontres de chacun. Les recettes qui en sont l’expression sont regroupées dans le chapitre « Famille ». Il y a les recettes héritées de Grand-Mère Saint-Pierre qui se faisait un point d’honneur de pratiquer la cuisine bourgeoise en toute rigueur suivant les canons de H.P. Pellaprat\footnote{Henri-Paul PELLAPRAT, L’Art Culinaire Moderne, 1952} et de réaliser avec la même rigueur les plats traditionnels des pays et régions où elle avait séjourné. Grand-Mère Ploudaniel, quant à elle, nous a laissé les goûteuses recettes qu’elle préparaient avec les produits du jardin, pommes de terre, échalotes, rhubarbe sans oublier les lapins de ses clapiers. Notre cuisine familiale s’est aussi enrichie de partage entre frères et sœurs et avec des amis fins cuisiniers et fins gourmets et, pour certains, de culture étrangère. Enfin, la nouvelle génération a introduit la notion de cuisine flexitarienne et donné le grade de « plat » aux légumes qui n’avaient jusque-là qu’un rôle de « garniture ». 
