Peut-on imaginer de servir un repas sans l’accompagner de boissons choisies pour en sublimer les saveurs ? Il est donc important de trouver les accords mets/vins les plus appropriés. Il faut remarquer que les goûts et les odeurs ne pouvant être qualibrés par une grandeur physique (contrairement aux couleurs et aux bruits), il n’existe pas de vocabulaire précis pour les qualifier et leur appréciation est essentiellement subjective.
Pour accorder les goûts des plats et des vins, il y a bien quelques principes de sommelier en la matière, mais la meilleure solution reste la dégustation et le partage des sensations entre convives.
Jean Paul a rassemblé dans sa cave des vins en provenance des différents bassins viticoles de France et quelques crus étrangers. Ces vins ont été découverts grâce à des dégustations entre amis ou à l’occasion de visites de salons.
Dans ce chapitre, les vins sont rangés par bassin (ou région). Chaque bassin est symbolisé par une icône qui apparait dans la vignette de couleur. Le symbole représente la forme du verre caractéristique du vignoble.
\medskip
{\renewcommand{\arraystretch}{1.7}
\begin{center}
\begin{tabular}{ l l l l }
\setbox0=\hbox{\put(0,0){\includegraphics[scale=0.021, trim= 0em -5em -5em -5em,]{Icones/icon_alsace_black.pdf}}}
	\parbox{\wd0}{\box0} 
	& \quad Alsace  & 
\setbox0=\hbox{\put(0,0){\includegraphics[scale=0.021, trim= 0em -5em -5em -5em,]{Icones/icon_bordeaux_black.pdf}}}
	\parbox{\wd0}{\box0}
	& \quad Bordeaux  \\ 
\setbox0=\hbox{\put(0,0){\includegraphics[scale=0.021, trim= 0em -5em -5em -5em,]{Icones/icon_bourgogne_black.pdf}}}
	\parbox{\wd0}{\box0}
	& \quad Bourgogne  & 
\setbox0=\hbox{\put(0,0){\includegraphics[scale=0.021, trim= 0em -5em -5em -5em,]{Icones/icon_champagne_black.pdf}}}
	\parbox{\wd0}{\box0}
	& \quad Champagne  \\ 
\setbox0=\hbox{\put(0,0){\includegraphics[scale=0.021, trim= 0em -5em -5em -5em,]{Icones/icon_jura_black.pdf}}}
	\parbox{\wd0}{\box0}
	& \quad Jura  & 
\setbox0=\hbox{\put(0,0){\includegraphics[scale=0.021, trim= 0em -5em -5em -5em,]{Icones/icon_languedoc_black.pdf}}}
	\parbox{\wd0}{\box0}
	& \quad Languedoc-Rousillon  \\ 
\setbox0=\hbox{\put(0,0){\includegraphics[scale=0.021, trim= 0em -5em -5em -5em,]{Icones/icon_provence_black.pdf}}}
	\parbox{\wd0}{\box0}
	& \quad Provence  & 
\setbox0=\hbox{\put(0,0){\includegraphics[scale=0.021, trim= 0em -5em -5em -5em,]{Icones/icon_sudouest_black.pdf}}}
	\parbox{\wd0}{\box0}
	& \quad Sud-Ouest \\ 
\setbox0=\hbox{\put(0,0){\includegraphics[scale=0.021, trim= 0em -5em -5em -5em,]{Icones/icon_loire_black.pdf}}}
	\parbox{\wd0}{\box0}
	& \quad Vallée de la Loire  & 
\setbox0=\hbox{\put(0,0){\includegraphics[scale=0.021, trim= 0em -5em -5em -5em,]{Icones/icon_rhone_black.pdf}}}
	\parbox{\wd0}{\box0}
	& \quad Vallée du Rhône  \\ 
\end{tabular}
\end{center}
}
\medskip
Un verre à vin est un verre à pied et doit être manipulé en le tenant par ce pied afin que la température du vin ne soit pas perturbée par la main. Par ailleurs, chaque vin doit être dégusté à une certaine température. Plus le vin doit rester frais, plus le pied du verre sera haut et le réservoir petit (Alsace, Loire, Champagne). Pour les vins qui ont en majorité des arômes légers qui s’évaporent vites (vins jeunes, Provence, Sud-Ouest, Languedoc-Roussillon), le bord du verre doit être resserré pour que le nez ait le temps de les apprécier. Pour les vins qui ont en majorité des arômes lourds (Jura), le bord sera large pour favoriser l’oxygénation qui fait remonter ces arômes. L’oxygénation sera aussi favorisée dans un verre à fond rond dans lequel on peut faire tourner le vin (vieux vins, Bordeaux, Rhône). Pour des vins plus complexes (Bourgogne), un bord resserré piègera les arômes légers et le fond rond permettra de faire remonter les arômes lourds.  
La carte des bassins viticoles de France est présentée ci-dessous.
\begin{figure}[!t]
\includegraphics[width=\textwidth]{/home/paugam/Dropbox/CarteVin/maps/bassinViticoleFrance.png}
\end{figure}
\newpage
Pour chaque bassin, sont ensuite répertoriés les châteaux ou domaines et leur proposition d’AOP (Appellations d’Origine Protégée qui remplacent les AOC, Appellation d’Origine Contrôlée) et d’IGP (Indications Géographiques Protégées). Leurs cépages sont rapportés lorsque le producteur les a précisés. Les coordonnées de ce dernier sont également indiquées.     
