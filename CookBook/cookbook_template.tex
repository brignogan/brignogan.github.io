\documentclass{article}
\usepackage{fancyhdr}
\usepackage{multicol}
\usepackage[utf8]{inputenc}
\usepackage[T1]{fontenc}
\usepackage{xfrac}    % Works better with other fonts
\usepackage{ulem}
\usepackage[french]{babel}
\usepackage{wrapfig}
\usepackage[%
    %a5paper,
    papersize={5.5in,8.5in},
    margin=0.75in,
    top=0.75in,
    bottom=0.75in,
    %twoside
    ]{geometry}

\usepackage{xcolor}
\usepackage{graphicx}

\usepackage{hyperref}
\hypersetup{
    colorlinks,
    citecolor=black,
    filecolor=black,
    linkcolor=black,
    urlcolor=black
}
\let\Sectionmark\sectionmark
\def\sectionmark#1{\def\Sectionname{#1}\Sectionmark{#1}}

\usepackage{sectsty}
\subsubsectionfont{\large}


\raggedcolumns
%\setlength{\multicolsep}{0pt}
%\setlength{\columnseprule}{1pt}

\makeatletter

\newif\if@mainmatter \@mainmattertrue

%% Borrowed from book.cls
\newcommand\frontmatter{%
    \cleardoublepage
  \@mainmatterfalse
  \pagenumbering{roman}}
\newcommand\mainmatter{%
    \cleardoublepage
  \@mainmattertrue
  \pagenumbering{arabic}}
\makeatother

%% Vary the colors at will

\definecolor{vegcolor}{rgb}{0,0.5,0.2}
\definecolor{frzcolor}{rgb}{0,0,1}
\definecolor{orangecolor}{rgb}{.83,.33,0}

% Your "recipes.sty" package starts here:
%% Thanks to alephzero for the excellent start:

\newcommand{\recipe}[2][\newpage]{%
	#1
	\lhead{\leftmark}%
    \chead{}%
    \rhead{}%
    \lfoot{}%
    \rfoot{}%
    \subsubsection{#2}%
}
%\newcommand{\serves}[2][Serves]{%
%    \chead{#1 #2}}
\newcommand{\vegetarian}{%
    \rhead{\large\color{vegcolor}\textbf{V}}}
\newcommand{\freeze}{%
    \lhead{\large\color{frzcolor}\textbf{F}}}
%% Optional arguments for alternate names for these:
%\newcommand{\preptime}[2][Prep time]{%
%    \lfoot{#1: #2}%
%}
%\newcommand{\cooktime}[2][Cook time]{%
%    \rfoot{#1: #2}%
%}
\newcommand{\temp}[1]{%
    $#1^\circ$C}

\newcommand{\preptime}[1]{
	\begingroup \setbox0=\hbox{\includegraphics[scale=0.04, trim= 0em -5em -5em -5em,]{temps_preparation_orange.pdf}}
				\parbox{\wd0}{\box0} temps preparation \quad {#1} \\ 
    \endgroup}
\newcommand{\cooktime}[1]{
	\begingroup \setbox0=\hbox{\includegraphics[scale=0.04, trim= 0em -5em -5em -5em,]{temps_cuisson_orange.pdf}}
				\parbox{\wd0}{\box0} temps cuisson \hspace{0.9cm} {#1} \\ 
    \endgroup}
\newcommand{\chilltime}[1]{
	\begingroup \setbox0=\hbox{\includegraphics[scale=0.04, trim= 0em -5em -5em -5em,]{temps_repos_orange.pdf}}
				\parbox{\wd0}{\box0} temps repos \hspace{1.2cm}   {#1} \\ 
    \endgroup}
\newcommand{\serve}[1]{
	\begingroup \setbox0=\hbox{\includegraphics[scale=0.035, trim= 0em -5em -5em -5em,]{people_281x281_orange.pdf}}
				\parbox{\wd0}{\box0} {#1} \\ 
    \endgroup}

\newcommand{\plogo}{\fbox{$\mathcal{PL}$}} % Generic dummy publisher logo

\newcommand{\showtime}[4]{
	\noindent
	\begin{minipage}[t]{\textwidth}
	\preptime{ #1 }
	\cooktime{ #2 }
	\chilltime{ #3 }
	\serve{ #4 }
	\end{minipage}		
}

%% Optional argument is the width of the graphic, default = 1in
\newcommand{\showit}[2]{%
    \noindent
	\begin{minipage}[t]{.38\textwidth}
	\vspace{0pt}
	\noindent
	\if###2##
    \else
    	#2
	\fi
  	\end{minipage}
	\hspace{.02\textwidth}
	\begin{minipage}[t]{.6\textwidth}
	\vspace{0pt}
	\includegraphics[width=\linewidth]{#1}%
  	\end{minipage}
}


%% Optional argument for a  heading within the ingredients section
\newcommand{\ingredients}[1][]{%
    \if###1##%
        {\Large\color{orangecolor}\textbf{Ingredients}}%
    \else
		\emph{#1} :%
    \fi
}


%% Use \obeylines to minimize markup
\newenvironment{ingreds}{%
    \parindent0pt
    \noindent
    \ingredients
    \par
    \medskip

	\setlength{\multicolsep}{0pt}
	\setlength{\columnseprule}{0.5pt}

    \begin{multicols}{2}
    \leftskip0em
    \rightskip0pt plus 3em
    \parskip=0.3em
    \obeylines
    \everypar={\hangindent1em}
	}{%
    \end{multicols}%
    \bigskip
}

\newcounter{stepnum}

%% Optional argument for an italicized pre-step
%% Also use obeylines to minimize markup here as well
\newcommand{\methods}[1][]{%
    \if###1##%
        \parindent0pt
    	\noindent
		{\Large\color{orangecolor}\textbf{Instructions}}%
		\par
		\medskip
    \else
        \noindent
     	\normalem
        \emph{#1}%
        \par
	\fi
}

\newenvironment{method}[1][]{%
    \setcounter{stepnum}{0}
    \begingroup
    \parindent0pt
    \parskip0.25em
	\leftskip2em
    \everypar={\llap{\stepcounter{stepnum}\hbox to2em{\thestepnum.\hfill}}}
}{%
    \par
    \endgroup
	
}
\newenvironment{method_noNumber}[1][]{%
    \setcounter{stepnum}{0}
    \begingroup
    \parindent0pt
    \parskip0.25em
	\leftskip2em
    \everypar={\llap{\hbox to2em{\hfill}}}
}{%
    \par
    \endgroup
	
}

%% Use \obeylines to minimize markup
\newenvironment{vin}[1][]{%
     \if###1##%
     \else
		\bigskip
		\begin{minipage}{\textwidth}
		\setcounter{stepnum}{0}
		\noindent
		{\color{orangecolor}\Large\textbf{Vin}}%
		\par
		\medskip
		\noindent
		{#1}
		\medskip
		\end{minipage}
	 \fi
}


%% Use \obeylines to minimize markup
\newenvironment{note}[1][]{%
     \if###1##%
     \else
		 \bigskip
		 \begin{minipage}{\textwidth}
		 \setcounter{stepnum}{0}
		 \noindent
		 {\color{orangecolor}\Large\textbf{Notes}}%
		 \par
		 \medskip
		 \noindent
		 {#1}
		 \end{minipage}
	 \fi
}


\pagestyle{fancy}
% End of "recipes.sty"


%%%%%%%%%%%%%%%%%%%%%%%%%%%%%%%%%%
\begin{document}
%%%%%%%%%%%%%%%%%%%%%%%%%%%%%%%%%%


\begin{titlepage} % Suppresses headers and footers on the title page
	
	\centering % Centre everything on the title page
	
	%------------------------------------------------
	%	Top rules
	%------------------------------------------------
	
	\rule{\textwidth}{1pt} % Thick horizontal rule
	
	\vspace{2pt}\vspace{-\baselineskip} % Whitespace between rules
	
	\rule{\textwidth}{0.4pt} % Thin horizontal rule
	
	\vspace{0.1\textheight} % Whitespace between the top rules and title
	
	%------------------------------------------------
	%	Title
	%------------------------------------------------
	\color{orangecolor}{	
	{\Huge Les Recettes}\\[0.5\baselineskip] % Title line 1
	{\Large de}\\[0.5\baselineskip] % Title line 2
	{\Huge Coat Tanguy} % Title line 3
	}
	\color{black}	
	\vspace{0.025\textheight} % Whitespace between the title and short horizontal rule
	
	\rule{0.3\textwidth}{0.4pt} % Short horizontal rule under the title
	
	\vspace{0.1\textheight} % Whitespace between the thin horizontal rule and the author name
	
	%------------------------------------------------
	%	Author
	%------------------------------------------------
	
	{\Large \textsc{Renee et Jean Paul Paugam}} % Author name
	
	\vfill % Whitespace between the author name and publisher
	
	%------------------------------------------------
	%	Publisher
	%------------------------------------------------
	
	{\large\color{orangecolor}{\plogo}}\\[0.5\baselineskip] % Publisher logo
	
	{\large\textsc{the publisher}} % Publisher
	
	\vspace{0.1\textheight} % Whitespace under the publisher text
	
	%------------------------------------------------
	%	Bottom rules
	%------------------------------------------------
	\color{black}
	\rule{\textwidth}{0.4pt} % Thin horizontal rule
	
	\vspace{2pt}\vspace{-\baselineskip} % Whitespace between rules
	
	\rule{\textwidth}{1pt} % Thick horizontal rule
	
\end{titlepage}

\clearpage % end title page
\begingroup
  \pagestyle{empty}
  \null
  \newpage
\endgroup

\frontmatter
\fancyhead{}
\fancyhead[EL,OL]{\leftmark}
\tableofcontents

\mainmatter

\section*{Avant Propos}
Nous avons rassemblé dans ces page toutes les recettes que nous avons mises en oeuvre pour garnir notre table au quotidien ou pour les jours de fête. Elles sont rangées dans les 4 rubriques: Famille, Bretagne, Maroc et Autriche. 

Ces plats sont issus de notre tradition familiale faite de cuisine française et de cuisine bretonne. Certains ont été rapportés de nos séjours à l’étranger : 13 années au Maroc pour toute la famille, 2 années à Vienne pour Maëlle, 4 années à Londres pour Ronan.

Pour chaque recette est indiqué le vin que Jean Paul a sorti de sa cave pour l’accompagner, cave constituée d’année en année avec l’aide d’amis fins gourmets.

Enfin, pour soigner sa table, on se reportera au livre “la cuisine du siècle“ de Catherine Bonnechere. C’est un livre de cuisine trouvé dans le grenier de Ploudaniel édité en 1899. Les chapîtres sur “le savoir vivre” et “les élégances de la table” sont irrésistibles! Elles sont visibles sur cette page internet\footnote{\url{https://en.calameo.com/read/0046395075f44e3515927}}.


XX

\end{document}
