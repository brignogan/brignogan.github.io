Chaque génération a connu ses émigrations. Pour des raisons professionnelles ou d’études, elles ont permis de découvrir d’autres cultures aux quatre coins de la France ou à l’étranger. Après l’Indochine, l’Algérie, les rives de la Méditerranée, la Picardie et l’Anjou pour les grands-parents, le Maroc pour les parents baby-boomers, les enfants de la génération Y se sont exilés dans les pays anglo-saxons (la Grande Bretagne et les Etats-Unis pour Ronan) ou germaniques (l’Autriche pour Maëlle).
Si Ronan a surtout choisi le rôle de critique gastronomique, Maëlle s’est essayée avec bonheur à la cuisine autrichienne. Etudiante ERASMUS à Vienne pendant 2 ans et très intéressée et douée pour la pâtisserie, outre un diplôme d’ingénieure, elle a rapporté de ce séjour les recettes regroupées dans le dernier chapitre « Autriche ». On y trouve certains plats traditionnels mais surtout les fameux gâteaux qui font la réputation des cafés viennois. Stefan Zweig écrit dans ses mémoires (Le Monde d’Hier) : « Le Kaffeehaus représente une institution d'un genre particulier, qui ne peut être comparée à aucune autre au monde. »
