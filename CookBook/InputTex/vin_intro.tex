Peut-on imaginer servir un repas sans l’accompagner de boissons choisies pour en sublimer les saveurs ? Il est donc important de trouver les accords mets/boissons les plus appropriés. Il faut remarquer que les goûts et les odeurs ne pouvant être qualibrés par une grandeur physique (contrairement aux couleurs et aux bruits), il n’existe pas de vocabulaire précis pour les qualifier et leur appréciation est essentiellement subjective.
Pour accorder les goûts des plats et des boissons, il y a bien quelques principes de sommelier en la matière, mais la meilleure solution reste la dégustation et le partage des sensations entre convives. 
Jean Paul a rassemblé dans sa cave des vins en provenance des différents bassins viticoles de France et quelques vins étrangers. Ces vins ont été découverts grâce à des dégustations entre amis ou à l’occasion de visites de salons. Pour les cidres, la sélection vient de la tradition familiale et ne retient donc que la production bretonne. 
Dans ce chapitre, les vins et cidres sont rangés par bassin (ou région). Chaque bassin est symbolisé par une icône qui apparait dans la vignette de couleur. Le symbole représente la forme du verre caractéristique du vignoble ou de la bolée pour le cidre.
\medskip
{\renewcommand{\arraystretch}{1.1}
\begin{center}
\begin{tabular}{ >{\centering\arraybackslash}p{0.18\linewidth}  >{\centering\arraybackslash}p{0.18\linewidth}  >{\centering\arraybackslash}p{0.18\linewidth}  >{\centering\arraybackslash}p{0.18\linewidth}}
\includegraphics[scale=0.021, trim= 0em -5em -5em -5em,]{Icones/icon_alsace_black.pdf}
&
\includegraphics[scale=0.021, trim= 0em -5em -5em -5em,]{Icones/icon_bordeaux_black.pdf}
&
\includegraphics[scale=0.021, trim= 0em -5em -5em -5em,]{Icones/icon_bourgogne_black.pdf}
&
\includegraphics[scale=0.021, trim= 0em -5em -5em -5em,]{Icones/icon_champagne_black.pdf}
\\
	Alsace  & Bordeaux & Bourgogne & Champagne \\
\end{tabular}
\end{center}
}	
{\renewcommand{\arraystretch}{1.1}
\begin{center}
%\begin{tabular}{ c c c c}
\begin{tabular}{ >{\centering\arraybackslash}p{0.18\linewidth}  >{\centering\arraybackslash}p{0.18\linewidth}  >{\centering\arraybackslash}p{0.18\linewidth}  >{\centering\arraybackslash}p{0.18\linewidth}}
\includegraphics[scale=0.021, trim= 0em 25em -5em -5em,]{Icones/icon_cidreB_black.pdf}
&
\includegraphics[scale=0.021, trim= 0em -5em -5em -5em,]{Icones/icon_jura_black.pdf}
&
\includegraphics[scale=0.021, trim= 0em -5em -5em -5em,]{Icones/icon_languedoc_black.pdf}
&
\includegraphics[scale=0.021, trim= 0em -5em -5em -5em,]{Icones/icon_provence_black.pdf}
\\
	\makecell{Cidre de\\Bretagne}  & Jura & Languedoc & Provence \\
\end{tabular}
\end{center}
}
{\renewcommand{\arraystretch}{1.1}
\begin{center}
%\begin{tabular}{ c c c c}
\begin{tabular}{ >{\centering\arraybackslash}p{0.18\linewidth}  >{\centering\arraybackslash}p{0.18\linewidth}  >{\centering\arraybackslash}p{0.18\linewidth}  >{\centering\arraybackslash}p{0.19\linewidth}}
\includegraphics[scale=0.021, trim= 0em -5em -5em -5em,]{Icones/icon_sudouest_black.pdf}
&
\includegraphics[scale=0.021, trim= 0em -5em -5em -5em,]{Icones/icon_loire_black.pdf}
&
\includegraphics[scale=0.021, trim= 0em -5em -5em -5em,]{Icones/icon_rhone_black.pdf}
&
\includegraphics[scale=0.021, trim= 0em -5em -5em -5em,]{Icones/icon_international_black.pdf}
\\
Sud Ouest  & \makecell{Vallée de\\la Loire}  & \makecell{Vallée du\\Rhône} & \makecell{Vin\\International} \\
\end{tabular}
\end{center}
}

%	& \quad Alsace  & 
%\setbox0=\hbox{\put(0,0){\includegraphics[scale=0.021, trim= 0em -5em -5em -5em,]{Icones/icon_bordeaux_black.pdf}}}
%	\parbox{\wd0}{\box0}
%	& \quad Bordeaux  \\ 
%\setbox0=\hbox{\put(0,0){\includegraphics[scale=0.021, trim= 0em -5em -5em -5em,]{Icones/icon_bourgogne_black.pdf}}}
%	\parbox{\wd0}{\box0}
%	& \quad Bourgogne  & 
%\setbox0=\hbox{\put(0,0){\includegraphics[scale=0.021, trim= 0em -5em -5em -5em,]{Icones/icon_champagne_black.pdf}}}
%	\parbox{\wd0}{\box0}
%	& \quad Champagne  \\ 
%\setbox0=\hbox{\put(0,0){\includegraphics[scale=0.021, trim= 0em -5em -5em -5em,]{Icones/icon_cidreB_black.pdf}}}
%	\parbox{\wd0}{\box0}
%	& \quad Cidre de Bretagne  & 
%\setbox0=\hbox{\put(0,0){\includegraphics[scale=0.021, trim= 0em -5em -5em -5em,]{Icones/icon_cidreN_black.pdf}}}
%	\parbox{\wd0}{\box0}
%	& \quad Cidre de Normandie  \\ 
%\setbox0=\hbox{\put(0,0){\includegraphics[scale=0.021, trim= 0em -5em -5em -5em,]{Icones/icon_jura_black.pdf}}}
%	\parbox{\wd0}{\box0}
%	& \quad Jura  & 
%\setbox0=\hbox{\put(0,0){\includegraphics[scale=0.021, trim= 0em -5em -5em -5em,]{Icones/icon_languedoc_black.pdf}}}
%	\parbox{\wd0}{\box0}
%	& \quad Languedoc-Rousillon  \\ 
%\setbox0=\hbox{\put(0,0){\includegraphics[scale=0.021, trim= 0em -5em -5em -5em,]{Icones/icon_provence_black.pdf}}}
%	\parbox{\wd0}{\box0}
%	& \quad Provence  & 
%\setbox0=\hbox{\put(0,0){\includegraphics[scale=0.021, trim= 0em -5em -5em -5em,]{Icones/icon_sudouest_black.pdf}}}
%	\parbox{\wd0}{\box0}
%	& \quad Sud-Ouest \\ 
%\setbox0=\hbox{\put(0,0){\includegraphics[scale=0.021, trim= 0em -5em -5em -5em,]{Icones/icon_loire_black.pdf}}}
%	\parbox{\wd0}{\box0}
%	& \quad Vallée de la Loire  & 
%\setbox0=\hbox{\put(0,0){\includegraphics[scale=0.021, trim= 0em -5em -5em -5em,]{Icones/icon_rhone_black.pdf}}}
%	\parbox{\wd0}{\box0}
%	& \quad Vallée du Rhône  \\ 
%\end{tabular}
%\end{center}
\medskip
Un verre à vin est un verre à pied et doit être manipulé en le tenant par ce pied afin que la température du vin ne soit pas perturbée par la main. Par ailleurs, chaque vin doit être dégusté à une certaine température. Plus le vin doit rester frais, plus le pied du verre sera haut et le réservoir petit (Alsace, Loire, Champagne). Pour les vins qui ont en majorité des arômes légers qui s’évaporent vites (vins jeunes, Provence, Sud-Ouest, Languedoc-Roussillon), le bord du verre doit être resserré pour que le nez ait le temps de les apprécier. Pour les vins qui ont en majorité des arômes lourds (Jura), le bord sera large pour favoriser l’oxygénation qui fait remonter ces arômes. L’oxygénation sera aussi favorisée dans un verre à fond rond dans lequel on peut faire tourner le vin (vieux vins, Bordeaux, Rhône). Pour des vins plus complexes (Bourgogne), un bord resserré piègera les arômes légers et le fond rond permettra de faire remonter les arômes lourds.
\begin{figure}[!t]
\includegraphics[width=\textwidth]{./VinMaps/bassinViticoleFrance.png}
\end{figure}

Par tradition, en Bretagne, on sert le cidre dans une bolée qui est en fait une tasse largement évasée. Mais pour apprécier le cidre, il est préférable, comme pour le champagne, de le servir dans un verre à pied.
  
La carte des bassins viticoles et cidricoles de France est présentée ci-contre.
Pour chaque bassin, sont ensuite répertoriés les châteaux ou domaines et leur proposition d’AOP (Appellations d’Origine Protégée qui remplacent les AOC, Appellation d’Origine Contrôlée) et d’IGP (Indications Géographiques Protégées). Leurs cépages sont rapportés lorsque le producteur les a précisés. Les coordonnées de ce dernier sont également indiquées.   
\vskip 32mm
\begin{figure}[!h]
\begin{center}
\includegraphics[width=.8\textwidth]{./SuppImg/vigne.png}
\end{center}
\end{figure}
