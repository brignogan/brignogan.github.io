La cuisine bretonne n’est pas faite que de crêpes, de kig-ha-fars ou de kouign-amann. Elle est bien plus riche et, avant tout, elle fait partie de notre culture et notre identité. Elle représente notre terroir parce qu’elle est le fruit de la rencontre des ressources naturelles, produits locaux agricoles et de la pêche, avec un peuple, ses us, ses coutumes et ses croyances. Il y a la cuisine des campagnes (argoat) et celle de la côte (armor), la cuisine de fête et celle de tous les jours. Si les recettes traditionnelles sont toujours scrupuleusement préparées, la nouvelle cuisine s’est aussi approprié ces ressources pour en faire des plats nouveaux.
Ce chapitre « Bretagne » est consacré aux recettes transmises de mère en fille et en fils à Saint-Pierre, à Ploudaniel et à Coat-Tanguy. Leur liste s’enrichit maintenant au gré des repas de famille ou de ceux partagés avec des amis bretons. Dans plusieurs de ces recettes, les ingrédients sont importants mais ce qui l’est encore plus ce sont les tours de main que Grand-Mère Saint-Pierre comme Grand-Mère Ploudaniel se faisaient un devoir et un plaisir de nous enseigner.
