La richesse de notre patrimoine culinaire doit beaucoup à nos rencontres avec des amis fins cuisiniers et fins gourmets et, pour certains, de culture étrangère. Il y a eu tant de bonheur à partager nos recettes et surtout à les mettre en oeuvre ensemble dans nos cuisines aux quatre coins de la France et d'ailleurs. Ce chapitre "Amis" regroupe celles que nous avons récupérées au cours de nos pérégrinations. Au Maroc, outre la cuisine locale, c'est au sein de la communauté coopérante que nous avons découvert les cuisines de différentes cultures. Grâce à Hanh et Jean Faucheu, nos amis originaires de Saïgon et Hanoï, c'est la cuisine vietnamienne qui nous a été apprise. C'est aussi à Rabat que Monique et Claude Fabas, nous ont offert les richesses de la cuisine de leur région d'origine, l'Occitanie. Durant nos années franciliennes, nous avons apprécié les recettes exotiques de notre amie Monique Salaün qui s'échappait de la vallée de Chevreuse par la pensée en s'adonnant à la cuisine étrangère. Quant à Cathy Broche, originaire du Taillan-Médoc et exilée aussi à Orsay, elle nous a initiés à la cuisine du Sud-Ouest. Enfin, des missions universitaires à Tomsk en Russie, nous avons rapporté les recettes des plats typiquement sibériens que Natalya Baranovskaya et Nina Osipova, nos collègues russes devenues de bonnes amies, nous ont fait découvrir au pays de Michel Strogoff.   