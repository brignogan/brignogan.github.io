La richesse de notre tradition culinaire doit beaucoup à nos rencontres avec des amis fins cuisiniers et fins gourmets et, pour certains, de culture étrangère. Il y a eu tant de bonheur à partager nos recettes et surtout à les mettre en oeuvre ensemble dans nos cuisines aux quatre coins de la France et d'ailleurs. C'est au Maroc que nous avons découvert la cuisine vietnamienne grâce à Hanh et Jean Faucheu, nos amis coopérants originaires de Saïgon. C'est aussi à Rabat que Monique et Claude Fabas, également coopérants, nous ont révélé les richesses de la cuisine de leur région d'origine, l'Occitanie. Durant nos années franciliennes, nous avons apprécié les recettes exotiques de notre amie Monique Salaün qui s'échappait de la vallée de Chevreuse par la pensée en s'adonnant à la cuisine étrangère. Quant à Cathy Broche, ... Enfin, Natalya Baranovskaya et Nina Osipova, nos collègues russes de Tomsk en Sibérie devenues de bonnes amies,...   