\section*{Avant Propos}
Les souvenirs de famille peuvent être une formidable source d’énergie. Ils ne sont pas un simple retour du passé. Ils constituent notre identité. Le socle commun des souvenirs est la trame des liens familiaux : les valeurs, les convictions, les histoires, les faits marquants du passé familial et les rites que tout le monde partage. Parmi ces rites, figurent en première place les recettes de cuisine.

Une recette de cuisine, ce n’est pas qu’une photo dans un livre ou sur un site internet accompagnée d’un algorithme qu’on veut, le plus souvent, vous faire passer pour le plus simple possible pour ne pas vous décourager \ldots~ Non, une recette de cuisine c’est tout d’abord la promesse de passer quelques heures entre l’évier, la plaque de cuisson, le four et le frigo à éplucher, découper, parer, laver, égoutter, essorer, mélanger, battre, cuire, remuer, goûter, présenter. C’est aussi, au bout du compte, le plaisir de régaler sa famille et ses amis. Si, de plus, ces recettes sont celles que l’on s’est passées de génération en génération, elles ont inévitablement le pouvoir de faire revenir des souvenirs que les odeurs et les saveurs peuvent réveiller. Tout le monde a sa madeleine\ldots

\textit{« Et tout d'un coup le souvenir m'est apparu. Ce goût c'était celui du petit morceau de madeleine que le dimanche matin, à Combray (parce que ce jour-là je ne sortais pas avant l'heure de la messe), quand j'allais lui dire bonjour dans sa chambre, ma tante Léonie m'offrait après l'avoir trempé dans son infusion de thé ou de tilleul. [\ldots] Quand d'un passé ancien rien ne subsiste, après la mort des êtres, après la destruction des choses, seules, plus frêles mais plus vivaces, plus immatérielles, plus persistantes, plus fidèles, l'odeur et la saveur restent encore longtemps, comme des âmes, à se rappeler, à attendre, à espérer, sur la ruine de tout le reste, à porter sans fléchir, sur leur gouttelette presque impalpable, l'édifice immense du souvenir. »} Marcel Proust, A la recherche du temps perdu, 1913

C’est Ronan qui a souhaité, dans un premier temps, réunir sur un site nos archives familiales (généalogie, chansons, poèmes, recettes)\footnote{\url{http://paugam.info/}}. Ensuite pour en faire plus profiter la famille et les amis, et en fin gastronome, il a décidé de faire une version imprimée des recettes. 
%Mettant à profit son expertise dans le codage informatique, il a créé le cadre du site puis la maquette de ce « cookbook » et joué le rôle du chef de rédaction.
      
Nous avons donc rassemblé dans ces pages toutes les recettes que nous mettons en œuvre pour garnir notre table au quotidien ou pour les jours de fête. Elles sont rangées dans 4 chapitres : Famille, Bretagne, Maroc et Autriche.

Le chapitre « Famille » regroupe les plats issus de notre tradition culinaire familiale. Si la cuisine de Saint-Pierre-Quilbignon ou celle de Ploudaniel ont été le lieu de nos premières armes, notre cuisine familiale s’est aussi enrichie au gré de nos rencontres.

Un chapitre « Bretagne » a été consacré aux recettes de cette région parce que, bretons, cette cuisine fait partie de notre culture, de notre identité. Préparés à partir des produits locaux de la terre et de la mer, certains de ces plats sont traditionnels et d’autres sont le fruit de la rencontre de la nouvelle cuisine avec notre terroir.

Le chapitre « Maroc » est le recueil des recettes de ce pays où nous avons passé 13 ans. Pendant ce séjour, grâce à Halima et Rkia, nos deux bonnes, nous avons pu apprécier cette cuisine riche, délicate et parfumée.

Le chapitre « Autriche » a été rédigé par Maëlle qui a passé 2 ans à Vienne. Elle a découvert la gastronomie autrichienne qui est une bonne synthèse de toutes les cuisines d’Europe Centrale et elle a surtout apprécié la pâtisserie viennoise. 

Un dernier chapitre est consacré à la cave que Jean Paul a constitué d’année en année avec l’aide d’amis, comme lui, amateurs de bons vins. Il assume le rôle de sommelier en cherchant les meilleurs accords mets et vins. Mais à notre table, pas de grands principes et de méthodes sophistiquées de dégustation, le vin est apprécié comme le produit tel qu’il est et à l’aune du plaisir qu’il permet de partager. Pour chaque recette, nous avons donc indiqué le choix de vins que Jean Paul a sélectionné dans sa cave pour accompagner le plat.

Nous vous souhaitons de belles aventures dans votre cuisine et un bon appétit en bonne compagnie.

\vspace{16pt}
\hfill
\parbox{2cm}{
Renée. 
}
