\section*{Avant Propos}
Les souvenirs de famille peuvent être une formidable source d’énergie. Ils ne sont pas un simple retour du passé. Ils constituent notre identité. Le socle commun des souvenirs est la trame des liens familiaux : les valeurs, les convictions, les histoires, les faits marquants du passé familial et les rites que tout le monde partage. Parmi ces rites, figurent en première place les recettes de cuisine. 

Une recette de cuisine, ce n’est pas qu’une photo dans un livre ou sur un site internet accompagnée d’un algorithme qu’on veut, le plus souvent, vous faire passer pour le plus simple possible pour ne pas vous décourager…. Non, une recette de cuisine c’est tout d’abord la promesse de passer quelques heures entre l’évier, la plaque de cuisson et le frigo à éplucher, découper, parer, laver, égoutter, essorer, mélanger, battre, cuire, remuer, goûter, présenter. C’est aussi, au bout du compte, le plaisir de régaler sa famille et ses amis. Si, de plus, ces recettes sont celles que l’on s’est passées de génération en génération, elles ont inévitablement le pouvoir de faire revenir des souvenirs que les odeurs et les saveurs peuvent réveiller. Tout le monde a sa madeleine \ldots{}

\textit{ << Et tout d'un coup le souvenir m'est apparu. Ce goût c'était celui du petit morceau de madeleine que le dimanche matin, à Combray (parce que ce jour-là je ne sortais pas avant l'heure de la messe), quand j'allais lui dire bonjour dans sa chambre, ma tante Léonie m'offrait après l'avoir trempé dans son infusion de thé ou de tilleul. [\ldots{}] Quand d'un passé ancien rien ne subsiste, après la mort des êtres, après la destruction des choses, seules, plus frêles mais plus vivaces, plus immatérielles, plus persistantes, plus fidèles, l'odeur et la saveur restent encore longtemps, comme des âmes, à se rappeler, à attendre, à espérer, sur la ruine de tout le reste, à porter sans fléchir, sur leur gouttelette presque impalpable, l'édifice immense du souvenir. >>}
Marcel Proust, A la recherche du temps perdu, 1913

C’est Ronan qui a souhaité, dans un premier temps, réunir sur un site\footnote{\url{http://paugam.info/}} nos archives familiales (généalogie, chansons, poèmes, recettes). Ensuite pour en faire plus profiter la famille et les amis, et en fin gastronome, il a décidé de faire une version imprimée des recettes. 
      
Nous avons donc rassemblé dans ces pages toutes les recettes que nous avons mises en œuvre pour garnir notre table au quotidien ou pour les jours de fête. Elles sont rangées dans 4 chapitres : Famille, Bretagne, Maroc et Autriche.

Le chapitre « Famille » regroupe les plats issus de notre tradition culinaire familiale. Beaucoup de recettes sont héritées de Grand-Mère Saint-Pierre qui se faisait un honneur de pratiquer la cuisine bourgeoise en toute rigueur suivant les canons de H.P. Pellaprat\footnote{Henri-Paul PELLAPRAT, L’Art Culinaire Moderne, 1952} et de réaliser avec la même rigueur les plats traditionnels des pays et régions où elle avait séjourné. Notre cuisine familiale s’est aussi enrichie de partage entre frère et sœurs et avec des amis fins cuisiniers et fins gourmets et pour certains de culture étrangère.

Un chapitre « Bretagne » a été consacré aux recettes de cette région parce que, bretons, cette cuisine fait partie de notre culture, de notre identité. Pour beaucoup d’entre elles, ces recettes ont été héritées de Grand-Mère Ploudaniel qui s’est toujours fait un grand plaisir de nous préparer ces plats traditionnels et nous enseigner ses tours de main. 

Le chapitre « Maroc » est le recueil des recettes de ce pays où nous avons passé 13 ans. Pendant ce séjour, nous avons apprécié cette cuisine riche, délicate et parfumée.

Le dernier chapitre « Autriche » a été rédigé par Maëlle qui a habité pendant 2 ans à Vienne. Très intéressée et douée pour la pâtisserie, elle en a ramené les recettes de certains plats traditionnels et surtout des gâteaux qui font la réputation des cafés viennois.
    
Pour chaque recette nous avons indiqué le choix de vins que Jean Paul a sélectionné dans sa cave pour accompagner le plat, cave constituée d'année en année avec l'aide d'amis fins gourmets. 

Nous vous souhaitons de belles aventures dans votre cuisine et un bon appétit en bonne compagnie.

\vspace{16pt}
\hfill
\parbox{2cm}{
Renée. 
}

\iffalse
Nous avons rassemblé dans ces page toutes les recettes que nous avons mises en oeuvre pour garnir notre table au quotidien ou pour les jours de fête. Elles sont rangées dans les $4$ rubriques: Famille, Bretagne, Maroc et Autriche. 

Ces plats sont issus de notre tradition familiale faite de cuisine française et de cuisine bretonne. Certains ont été rapportés de nos séjours à l’étranger : $13$ années au Maroc pour toute la famille, $2$ années à Vienne pour Maëlle, $4$ années à Londres pour Ronan.

Pour chaque recette est indiqué le vin que Jean Paul a sorti de sa cave pour l’accompagner, cave constituée d’année en année avec l’aide d’amis fins gourmets.

Enfin, pour soigner sa table, on se reportera au livre “la cuisine du siècle“ de Catherine Bonnechere. C’est un livre de cuisine trouvé dans le grenier de Ploudaniel édité en $1899$. Les chapîtres sur “le savoir vivre” et “les élégances de la table” sont irrésistibles! Elles sont visibles sur cette page internet\footnote{\url{https://en.calameo.com/read/0046395075f44e3515927}}.
\fi

