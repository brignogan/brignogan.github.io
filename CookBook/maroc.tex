Le Maroc est le pays où nous avons passé 13 ans. Pendant ce séjour, nous avons eu deux bonnes, Halima et Rkia. Grâce à elles, nous avons réussi à nous adapter aux us et coutumes de ce pays et en comprendre les subtilités. Halima s’occupa de notre maison, à Rabat, de 1975 à 1980. C’était une vieille dame délicieuse qui avait travaillé dans les cuisines d’un ministère et qui était donc une excellente cuisinière. Rkia, quant à elle, partagea notre quotidien de 1980 à 1988, à Rabat après le décès de Halima, puis à Agadir. Elle était toute jeune, très curieuse et prête à toutes les expériences culinaires.
Ce chapitre « Maroc » est donc le recueil de leurs recettes. Elles nous ont préparé tous ces plats avec tant de plaisir et de générosité. Pendant ces treize années, elles ont eu le souci de nous faire découvrir toutes les richesses de la cuisine marocaine si parfumée et délicate. Aujourd’hui, les odeurs et le goût d’un « tagine » nous rappellent l’ambiance des souks, des repas pantagruéliques chez les collègues marocains, des restaurants de palace comme de ceux du fin fond du bled.
